%----------------------------------------------------------------------------------------
%	problem set 6 phys 262 due 03 Dec 2015 (or actually before final time?)
%----------------------------------------------------------------------------------------
%%%%%%%%%%%%%%%%%%%%%%%%%%%%%%%%%%%%%%%%%
% Short Sectioned Assignment
% LaTeX Template
% Version 1.0 (5/5/12)
%
% This template has been downloaded from:
% http://www.LaTeXTemplates.com
%
% Original author:
% Frits Wenneker (http://www.howtotex.com)
%
% License:
% CC BY-NC-SA 3.0 (http://creativecommons.org/licenses/by-nc-sa/3.0/)
%
%%%%%%%%%%%%%%%%%%%%%%%%%%%%%%%%%%%%%%%%%

%----------------------------------------------------------------------------------------
%	PACKAGES AND OTHER DOCUMENT CONFIGURATIONS
%----------------------------------------------------------------------------------------

\documentclass[11pt, letterpaper]{scrartcl} % letter paper and 11pt font size

\usepackage[T1]{fontenc} % Use 8-bit encoding that has 256 glyphs
% \usepackage{fourier} % Use the Adobe Utopia font for the document - comment this line to return to the LaTeX default
\usepackage[english]{babel} % English language/hyphenation
\usepackage{amsmath,amsfonts,amsthm,bm} % Math packages (bm added by victor)

\usepackage{listings}           % for code inclusion (added by victor)
\lstset{basicstyle=\footnotesize\ttfamily}

\usepackage{lipsum} % Used for inserting dummy 'Lorem ipsum' text into the template

\usepackage{sectsty} % Allows customizing section commands
\allsectionsfont{\normalfont \bfseries} % Make all sections centered, the default font and small caps (\centering, \scshape removed)

\usepackage{fancyhdr} % Custom headers and footers

\usepackage{tcolorbox} % added by victor, for adding boxes
\usepackage{graphicx}

\pagestyle{fancyplain} % Makes all pages in the document conform to the custom headers and footers
\fancyhead{} % No page header - if you want one, create it in the same way as the footers below
\fancyfoot[L]{} % Empty left footer
\fancyfoot[C]{} % Empty center footer
\fancyfoot[R]{\thepage} % Page numbering for right footer
\renewcommand{\headrulewidth}{0pt} % Remove header underlines
\renewcommand{\footrulewidth}{0pt} % Remove footer underlines
\setlength{\headheight}{0pt} % Customize the height of the header

\numberwithin{equation}{section} % Number equations within sections (i.e. 1.1, 1.2, 2.1, 2.2 instead of 1, 2, 3, 4)
\numberwithin{figure}{section} % Number figures within sections (i.e. 1.1, 1.2, 2.1, 2.2 instead of 1, 2, 3, 4)
\numberwithin{table}{section} % Number tables within sections (i.e. 1.1, 1.2, 2.1, 2.2 instead of 1, 2, 3, 4)

\setlength\parindent{0pt} % Removes all indentation from paragraphs - comment this line for an assignment with lots of text

\allowdisplaybreaks             % allow page breaks between long math sections

\renewcommand{\thesubsection}{\thesection.\alph{subsection}} % enumerate subsections by letter

\newcommand\numberthis{\addtocounter{equation}{1}\tag{\theequation}}
\renewcommand\vec{\mathbf}      % for more legibility

\def\bbar{{\mathchar'26\mkern-8mu b}} 

\DeclareMathOperator{\Haml}{\mathcal{H}}
\DeclareMathOperator{\Tr}{\text{Tr}}

%----------------------------------------------------------------------------------------
%	TITLE SECTION
%----------------------------------------------------------------------------------------

\newcommand{\horrule}[1]{\rule{\linewidth}{#1}} % Create horizontal rule command with 1 argument of height

\title{	
\normalfont \normalsize 
\textsc{Harvard Physics 262: Statistical Physics - Problem Set 6} \\ [20pt] % Your university, school and/or department name(s)
% \horrule{0.5pt} \\[0.4cm] % Thin top horizontal rule
% \large Problem Set 6 \\ % The assignment title
% \horrule{2pt} \\[0.5cm] % Thick bottom horizontal rule
}

\author{Rostam Razban & Yuanchi Zhao} % Your name

\date{\normalsize 03 December 2015} % Today's date or a custom date

\begin{document}

%\maketitle % Print the title

% 1 ------------------------------------------------------------------------------------------
\section{Metropolis Simulation}

% a --------------------------------------------------
\subsection{MCMC transition matrix}
The transition matrix W is written as
\begin{align}
W(C\rightarrow C^{'}) &= e^{-\beta (E(C^{'})-E(C))} \label{tmat}\\
W(C^{'}\rightarrow C) &= 1
\end{align}
when $E(C^{'})>E(C)$

\subsubsection{Prove ergodicity}
For ergodicity, it suffices to show that there exists one stationary probability.
\begin{align}
P_{st}(C^{'})=\sum\limits_{C} P_{st}(C)W(C\rightarrow C^{'}) 
\end{align}
Plugging in \ref{tmat}, one obtains
\begin{align}
P_{st}(C^{'})&=e^{-\beta E(C^{'})}\sum\limits_{C}  P_{st}(C) e^{\beta E(C)}\\
P_{st}(C^{'})&=e^{-\beta E(C^{'})}\alpha
\end{align}
where $\alpha$ is just a constant.
\subsubsection{Prove detailed balance w.r.t. equilibrium weight}
\begin{align}
\dfrac{W(C\rightarrow C^{'})}{W(C^{'}\rightarrow C)} = \dfrac{e^{-\beta (E(C^{'})-E(C))}}{1} \label{ratio}
\end{align}
Rearranging \ref{ratio}, one obtains
\begin{align}
e^{-\beta E(C)}W(C\rightarrow C^{'}) = e^{-\beta E(C^{'})}W(C^{'}\rightarrow C)
\end{align}

% b --------------------------------------------------
\subsection{Metropolis algorithm implementation}

The code, ising\_MCMC.py, is appended. For system sizes, $4x4$, $8x8$,
and $32x32$, 2500-sweep runs were run at 1.5, 2.0, 2.2, 2.3, 2.6 and
3.0 kT (coupling constant $J=1$). Plots of $E(\tau)$ and $M(\tau)$ are
attached. To run the program:

\begin{lstlisting}
  for size in 4 8 32; do
      for temperature in 1.5 2.0 2.2 2.3 2.6 3.0; do
          ./ising_MCMC.py $size $temperature --verbose \
              --out-pkl 2500_sweeps_${size}x${size}_${temperature}kT.pkl \
              | tee 2500_sweeps_${size}x${size}_${temperature}kT.log
      done
  done
\end{lstlisting}


% c --------------------------------------------------
\subsection{Simulation results}

$\chi$ in the limit of high temperatures. Using Mathematica,


% d --------------------------------------------------
\subsection{Critical exponents}


% e --------------------------------------------------
\subsection{Finite size scaling}


% 2 ------------------------------------------------------------------------------------------
\section{Swendsen-Ma MCRG}


 % ------------------------------------------------------------------------------------------
\end{document}
