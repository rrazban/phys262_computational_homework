%----------------------------------------------------------------------------------------
%	problem set 6 phys 262 due 03 Dec 2015 (or actually before final time)
%----------------------------------------------------------------------------------------
%%%%%%%%%%%%%%%%%%%%%%%%%%%%%%%%%%%%%%%%%
% Short Sectioned Assignment
% LaTeX Template
% Version 1.0 (5/5/12)
%
% This template has been downloaded from:
% http://www.LaTeXTemplates.com
%
% Original author:
% Frits Wenneker (http://www.howtotex.com)
%
% License:
% CC BY-NC-SA 3.0 (http://creativecommons.org/licenses/by-nc-sa/3.0/)
%
%%%%%%%%%%%%%%%%%%%%%%%%%%%%%%%%%%%%%%%%%

%----------------------------------------------------------------------------------------
%	PACKAGES AND OTHER DOCUMENT CONFIGURATIONS
%----------------------------------------------------------------------------------------

\documentclass[11pt, letterpaper]{scrartcl} % letter paper and 11pt font size

\usepackage[T1]{fontenc} % Use 8-bit encoding that has 256 glyphs
% \usepackage{fourier} % Use the Adobe Utopia font for the document - comment this line to return to the LaTeX default
\usepackage[english]{babel} % English language/hyphenation
\usepackage{amsmath,amsfonts,amsthm,bm} % Math packages (bm added by victor)

\usepackage{listings}           % for code inclusion (added by victor)

\usepackage{lipsum} % Used for inserting dummy 'Lorem ipsum' text into the template

\usepackage{sectsty} % Allows customizing section commands
\allsectionsfont{\normalfont \bfseries} % Make all sections centered, the default font and small caps (\centering, \scshape removed)

\usepackage{fancyhdr} % Custom headers and footers

\usepackage{tcolorbox} % added by victor, for adding boxes
\usepackage{graphicx}

\pagestyle{fancyplain} % Makes all pages in the document conform to the custom headers and footers
\fancyhead{} % No page header - if you want one, create it in the same way as the footers below
\fancyfoot[L]{} % Empty left footer
\fancyfoot[C]{} % Empty center footer
\fancyfoot[R]{\thepage} % Page numbering for right footer
\renewcommand{\headrulewidth}{0pt} % Remove header underlines
\renewcommand{\footrulewidth}{0pt} % Remove footer underlines
\setlength{\headheight}{0pt} % Customize the height of the header

\numberwithin{equation}{section} % Number equations within sections (i.e. 1.1, 1.2, 2.1, 2.2 instead of 1, 2, 3, 4)
\numberwithin{figure}{section} % Number figures within sections (i.e. 1.1, 1.2, 2.1, 2.2 instead of 1, 2, 3, 4)
\numberwithin{table}{section} % Number tables within sections (i.e. 1.1, 1.2, 2.1, 2.2 instead of 1, 2, 3, 4)

\setlength\parindent{0pt} % Removes all indentation from paragraphs - comment this line for an assignment with lots of text

\allowdisplaybreaks             % allow page breaks between long math sections

\renewcommand{\thesubsection}{\thesection.\alph{subsection}} % enumerate subsections by letter

\newcommand\numberthis{\addtocounter{equation}{1}\tag{\theequation}}
\renewcommand\vec{\mathbf}      % for more legibility

\def\bbar{{\mathchar'26\mkern-8mu b}} 

\DeclareMathOperator{\Haml}{\mathcal{H}}
\DeclareMathOperator{\Tr}{\text{Tr}}

%----------------------------------------------------------------------------------------
%	TITLE SECTION
%----------------------------------------------------------------------------------------

\newcommand{\horrule}[1]{\rule{\linewidth}{#1}} % Create horizontal rule command with 1 argument of height

\title{	
\normalfont \normalsize 
\textsc{Harvard Physics 262: Statistical Physics - Problem Set 6} \\ [20pt] % Your university, school and/or department name(s)
% \horrule{0.5pt} \\[0.4cm] % Thin top horizontal rule
% \large Problem Set 2 \\ % The assignment title
% \horrule{2pt} \\[0.5cm] % Thick bottom horizontal rule
}

\author{Yuanchi ``Victor'' Zhao} % Your name

\date{\normalsize 03 December 2015} % Today's date or a custom date

\begin{document}

\maketitle % Print the title

% 1 ------------------------------------------------------------------------------------------
\section{Metropolis Simulation}

% --------------------------------------------------
\subsection{MCMC transition matrix}

The induced magnetic moment is the sum of the moment of each atom
projected in the direction of $H$. So the approach is to obtain $M =
N*\mu \langle \cos \alpha \rangle$. The contribution of the positions
and momenta to the partition function is separable:

\begin{align}
  Q(N,V,T) &= \frac{1}{h^{3N}} \int d^N\vec{p} \int d^N\vec{q} \exp(-\beta \bar{\Haml}) Z \\
  &= \frac{V^N}{N! \lambda^{3N}} Z(N,H,T)\\
  \lambda &= \sqrt{ \frac{\beta h^2}{2\pi m} }
\end{align}

where our magnetic partition function, $Z$, is obtained by integrating
all rotational orientations of the magnetic moment of each atom.

\begin{align}
  Z = \left[ \int_0^{2\pi} d\phi \int_0^\pi d\alpha \, \mu^2 \sin (\alpha) \exp( +\beta\mu H \cos \alpha) \right]^N
\end{align}

Because there are no interaction terms between atoms, we can calculate
the expectation value of $\cos \alpha$ for one atom, and multiply by
$N\mu$ to obtain the total induced magnetic moment.

\begin{align}
  \langle \cos \alpha \rangle &= \frac{\int_0^{2\pi} d\phi \int_0^\pi d\alpha \, \mu^2 \cos(\alpha) \sin (\alpha) \exp( \beta\mu H \cos \alpha)} {Z_1} \\
                              &= \frac{ \int_0^{2\pi} d\phi \int_0^\pi d\alpha \, \mu^2 \cos(\alpha) \sin(\alpha) \exp( \beta\mu H \cos \alpha)}
                                {\int_0^{2\pi} d\phi \int_0^\pi d\alpha \, \mu^2 \sin (\alpha) \exp( \beta\mu H \cos \alpha)} \\
                              &= \frac{\int_0^\pi d\alpha \, \cos(\alpha) \sin(\alpha) \exp( \beta\mu H \cos \alpha)}
                                {\int_0^\pi d\alpha \, \sin (\alpha) \exp( \beta\mu H \cos \alpha)} \\
                              &= \frac{-1}{\beta\mu H} \frac{\int_0^\pi d\alpha \, \cos(\alpha) \frac{d}{d\alpha} \exp( \beta\mu H \cos \alpha)}
                                {\int_0^\pi d\alpha \, \sin (\alpha) \exp( \beta\mu H \cos \alpha)} \\
                              &= \frac{-1}{\beta\mu H} \left[ \frac{ \cos (\alpha) \exp(\beta\mu H \cos \alpha) |_0^\pi + \int_0^\pi d\alpha \, \sin(\alpha) \exp( \beta\mu H \cos \alpha)}
                                {\int_0^\pi d\alpha \, \sin (\alpha) \exp( \beta\mu H \cos \alpha)} \right] \\
                              &= \frac{-1}{\beta\mu H} \frac{\left( -\exp(-\beta\mu H) - \exp(\beta\mu H)  \right)}
                                {\int_0^\pi d\alpha \, \sin (\alpha) \exp( \beta\mu H \cos \alpha)} - \frac{1}{\beta\mu H} \\
                              &= \frac{1}{\beta\mu H} \frac{\left( \exp(-\beta\mu H) + \exp(\beta\mu H)  \right)}
                                {\frac{1}{-\beta\mu H} \int_0^\pi d\alpha \frac{d}{d\alpha} \exp( \beta\mu H \cos \alpha)} - \frac{1}{\beta\mu H} \\
                              &= \frac{-\left( \exp(-\beta\mu H) + \exp(\beta\mu H)  \right)}
                                {\exp( \beta\mu H \cos \alpha) |_0^\pi } - \frac{1}{\beta\mu H} \\
                              &= \frac{-\left( \exp(-\beta\mu H) + \exp(\beta\mu H)  \right)}
                                {\left( \exp(-\beta\mu H) - \exp(\beta\mu H) \right)} - \frac{1}{\beta\mu H} \\
                              &= \frac{\left( \exp(\beta\mu H) + \exp(-\beta\mu H)  \right)}
                                {\left( \exp(\beta\mu H) - \exp(-\beta\mu H) \right)} - \frac{1}{\beta\mu H} \\
                              &= \coth (\beta\mu H) - \frac{1}{\beta\mu H}
\end{align}

Where we've used integration by parts. So therefore

\begin{align}
  M &= N\mu \langle {\cos \alpha} \rangle\\
  M &= N\mu(\coth \theta - \frac{1}{\theta})\\
  \theta &= \beta \mu H = \frac{\beta\mu}{kT}
\end{align}

% --------------------------------------------------
\subsection{Metropolis algorithm implementation}

The code, ising_MCMC.py, is appended. For system sizes, $4x4$, $8x8$,
and $32x32$, 2500-sweep runs were run at 1.5, 2.0, 2.2, 2.3, 2.6 and
3.0 kT (coupling constant $J=1$). Plots of $E(\tau)$ and $M(\tau)$ are
attached. To run the program:

\begin{lstlisting}
  for size in 4 8 32; do
      for temperature in 1.5 2.0 2.2 2.3 2.6 3.0; do
          ./ising_MCMC.py $size $temperature --verbose \
              --out-pkl 2500_sweeps_${size}x${size}_${temperature}kT.pkl \
              | tee 2500_sweeps_${size}x${size}_${temperature}kT.log
      done
  done
\end{lstlisting}


% --------------------------------------------------
\subsection{SImulation results}

$\chi$ in the limit of high temperatures. Using Mathematica,

\begin{align}
  \lim_{T->\infty} &\left[\frac{k^2T^2}{\mu^2H^2} - \text{csch}^2(\frac{\mu H}{kT})\right] = 1/3\\
                  \chi &= \frac{\mu^2 }{3kT}\\
                         C &= \frac{\mu^2}{3k}
\end{align}

% --------------------------------------------------
\subsection{Critical exponents}


% --------------------------------------------------
\subsection{Finite size scaling}


% 2 ------------------------------------------------------------------------------------------
\section{Swendsen-Ma MCRG}


 % ------------------------------------------------------------------------------------------
\end{document}
